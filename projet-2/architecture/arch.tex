\documentclass{scrartcl}
% Language
\usepackage[utf8]{inputenc}
\usepackage[T1]{fontenc}      
\usepackage[francais]{babel}
% Links
\usepackage{url}
\usepackage{hyperref}
\hypersetup{
        colorlinks,
        citecolor=black,
        filecolor=black,
        linkcolor=black,
        urlcolor=black
}

\title{LSINF1252 - Factorisation de nombres}
\author{\textsc{Monnoyer} Charles et \textsc{Paris} Antoine}
\date{\today}

\begin{document}
\maketitle

\section{Architecture globale}
Pour structurer une application qui réalise des calculs,
il est courant d'utiliser des producteurs/consommateurs\cite{syll}.
Dans notre cas, les producteurs seraient chargé d'extraire
les nombres à factoriser des fichiers passés en ligne
de commande et de les placer dans le buffer. Les
consommateurs seraint quant à eux chargé de factoriser
les nombres contenus dans le buffer et de sauvegarder
le résulat dans une structure de données adéquates. 

\section{Threads utilisés}

\section{Mécanismes de synchronisation}

\section{Principale structures de données}

\section{Algorithme de factorisation}
L'algorithme de factorisation choisi est l'algorithme
\textsc{Pollard-Rho}. Cet algorithme est très efficace
et a pour avantage d'utiliser un nombre constant
d'emplacement en mémoire\cite{algo}.

\bibliographystyle{plain}
\bibliography{arch.bib}

\end{document}