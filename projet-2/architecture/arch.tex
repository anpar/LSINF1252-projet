\documentclass{scrartcl}
% Language
\usepackage[utf8]{inputenc}
\usepackage[T1]{fontenc}      
\usepackage[francais]{babel}
% Links
\usepackage{url}
\usepackage{hyperref}
\hypersetup{
        colorlinks,
        citecolor=black,
        filecolor=black,
        linkcolor=black,
        urlcolor=black
}

\title{LSINF1252 - Factorisation de nombres}
\author{\textsc{Monnoyer} Charles et \textsc{Paris} Antoine}
\date{\today}

\begin{document}
\maketitle

\section{Architecture globale}
Pour structurer une application qui réalise des calculs,
il est courant d'utiliser des producteurs/consommateurs\cite{syll}.
Dans notre cas, les producteurs seraient chargé d'extraire
les nombres à factoriser des fichiers passés en ligne
de commande et de les placer dans le buffer. Les
consommateurs seraint quant à eux chargé de factoriser
les nombres contenus dans le buffer et de sauvegarder
le résulat dans une structure de données adéquates. 

\section{Threads utilisés}

\section{Mécanismes de synchronisation}

\section{Principale structures de données}

\section{Algorithme de factorisation}
L'algorithme que nous avons décidé d'implémenter pour ce projet
est un algorithme à but général (c'est à dire dont le temps
d'éxécution dépend de la taille du nombre à factoriser, et
non de la taille de ces facteurs premiers). Il s'agit du
\textit{Shanks's square forms factorization algorithm (SQUFOF)}.
Nous avons choisis cet algorithme car il possède une bonne
complexité temporelle ($\sqrt[4]{n}$, où $n$ est le nombre à
factorisé) tout en étant facile à implémenter. Deux contraintes
importantes sont cependant à noter, cet algorithme ne fonctionne
pas si son entrée $n$ est un carré parfait ou un nombre premier. 
Cependant, cela ne posera pas problème en pratique.
En effet, dans le cas où $n$ est un carré parfait, il suffit de
donner $\sqrt{n}$ en entrée à l'algorithme\footnote{Les facteurs
premiers de $n$ sont identiques aux facteurs premiers de $\sqrt{n}$.}
Dans le cas où $n$ est un nombre premier, l'algorithme est inutile
et il peut simplement retourner $n$.

\bibliographystyle{plain}
\bibliography{arch.bib}

\end{document}