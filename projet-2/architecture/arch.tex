\documentclass{scrartcl}
% Language
\usepackage[utf8]{inputenc}
\usepackage[T1]{fontenc}      
\usepackage[francais]{babel}
% Links
\usepackage{url}
\usepackage{hyperref}
\hypersetup{
        colorlinks,
        citecolor=black,
        filecolor=black,
        linkcolor=black,
        urlcolor=black
}

\title{LSINF1252 - Factorisation de nombres}
\author{\textsc{Monnoyer} Charles et \textsc{Paris} Antoine}
\date{\today}

\begin{document}
\maketitle

\section{Architecture globale}
Pour structurer une application qui réalise des calculs,
il est courant d'utiliser des producteurs/consommateurs\cite{syll}.
Pour cette application, il est peut-être même plus intéressant
d'utiliser deux buffers et donc d'utiliser des producteurs/
consommateurs-producteurs/consommateurs.

Dans notre cas, les producteurs seraient chargé d'extraire
les nombres à factoriser des fichiers passés en ligne
de commande et de les placer dans le premier buffer. Les
consommateurs-producteurs seraient quant à eux chargé de
factoriser les nombres contenus dans le premier buffer pour
ensuite insérer les facteurs premiers de ces nombres
dans le second buffer. Enfin, les consommateurs
permettrait de vider le second buffer pour stocker les
facteurs premiers dans une structure données (qui sera
discutée en section \ref{sec:data-structure}.

%\begin{figure}
%	\centering
%	\includegraphics[scale=0.5]{img/global.png}
%	\caption{Schéma de notre architecture.}
%	\label{fig:global}
%\end{figure}

\section{Threads utilisés}
Pour bien fonctionner, notre application nécessite donc
aux minimum 3 threads : un extracteur, un calculateur et
un sauveur de résultats.

\section{Mécanismes de synchronisation}
Pour implémenter un simple problème de producteurs-
consommateurs, il faut utiliser un mutex pour protéger
le buffer et deux sémaphores pour gérer les producteurs/
consommateurs. Dans ce cas-ci, il faudra utiliser deux
mutex pour protéger les deux buffers et quatre sémaphores.

\section{Principale structures de données}
\label{sec:data-structure}
Les deux buffers seront représentés par des tableaux d'entiers.
% Il reste à déterminer les tailles relatives des deux buffers. 
% D'après moi le deuxième devrait être plus grand.
Les facteurs premiers seront ensuite déplacé du second
buffer vers une liste châinée de structure.
Ces structures contiendront un facteur premier et le
nombre d'occurence associé à ce facteur.

\section{Algorithme de factorisation}
L'algorithme que nous avons décidé d'implémenter pour ce projet
est un algorithme à but général (c'est à dire dont le temps
d'éxécution dépend de la taille du nombre à factoriser, et
non de la taille de ces facteurs premiers). Il s'agit du
\textit{Shanks's square forms factorization algorithm (SQUFOF)}.
Nous avons choisis cet algorithme car il possède une bonne
complexité temporelle ($\sqrt[4]{n}$, où $n$ est le nombre à
factorisé) tout en étant facile à implémenter. Deux contraintes
importantes sont cependant à noter, cet algorithme ne fonctionne
pas si son entrée $n$ est un carré parfait ou un nombre premier. 
Cependant, cela ne posera pas problème en pratique.
En effet, dans le cas où $n$ est un carré parfait, il suffit de
donner $\sqrt{n}$ en entrée à l'algorithme\footnote{Les facteurs
premiers de $n$ sont identiques aux facteurs premiers de $\sqrt{n}$.}
Dans le cas où $n$ est un nombre premier, l'algorithme est inutile
et il peut simplement retourner $n$.

\bibliographystyle{plain}
\bibliography{arch.bib}

\end{document}